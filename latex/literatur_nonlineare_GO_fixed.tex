
\begin{filecontents*}{refs.bib}
@misc{dumas2005nonlinear,
  author = {Dumas, Éric},
  title  = {About nonlinear geometric optics},
  year   = {2005},
  eprint = {math/0511629},
  archivePrefix = {arXiv},
  primaryClass = {math.AP}
}

@article{couairon2007femtosecond,
  author  = {Couairon, Arnaud and Mysyrowicz, André},
  title   = {Femtosecond filamentation in transparent media},
  journal = {Physics Reports},
  year    = {2007},
  volume  = {441},
  pages   = {47--189}
}

@article{kolesik2004nonlinear,
  author  = {Kolesik, Miroslav and Moloney, Jerome V.},
  title   = {Nonlinear optical pulse propagation simulation: From Maxwell’s to unidirectional equations},
  journal = {Physical Review E},
  year    = {2004},
  volume  = {70},
  number  = {3},
  pages   = {036604}
}

@article{berge2007ultrashort,
  author  = {Bergé, Luc and Skupin, Stefan and Nuter, Romain and Kasparian, Jér{\^o}me and Wolf, Jean-Pierre},
  title   = {Ultrashort filaments of light in weakly ionized, optically transparent media},
  journal = {Reports on Progress in Physics},
  year    = {2007},
  volume  = {70},
  number  = {10},
  pages   = {1633--1713}
}

@article{brabec1997singlecycle,
  author  = {Brabec, Thomas and Krausz, Ferenc},
  title   = {Nonlinear Optical Pulse Propagation in the Single-Cycle Regime},
  journal = {Physical Review Letters},
  year    = {1997},
  volume  = {78},
  number  = {17},
  pages   = {3282--3285}
}

@book{agrawal2012nonlinear,
  author    = {Agrawal, Govind P.},
  title     = {Nonlinear Fiber Optics},
  edition   = {5},
  year      = {2012},
  publisher = {Academic Press}
}

% ==== Neuere Arbeiten (2020–2024) ====
@article{lassonde2020temporalconvolution,
  author  = {Lassonde, Philippe and Mitautoren},
  title   = {Information transfer via temporal convolution in nonlinear optics},
  journal = {Scientific Reports},
  year    = {2020}
}

@article{gliserin2022penguin,
  author  = {Gliserin, Alexey und Mitautoren},
  title   = {Complete characterization of ultrafast optical fields by Phase-Enabled Nonlinear Gating with Unbalanced Intensity (PENGUIN)},
  journal = {Light: Science and Applications},
  year    = {2022}
}

@article{cai2020temporalmeasurements,
  author  = {Cai, Yi und Mitautoren},
  title   = {The Development of the Temporal Measurements for Ultrashort Laser Pulses},
  journal = {Applied Sciences},
  year    = {2020}
}

@article{iordanova2024broadband,
  author  = {Iordanova, Elena},
  title   = {Linear and Nonlinear Optics of Broad-Band Laser Pulses},
  journal = {ACS Omega},
  year    = {2024}
}
\end{filecontents*}

\documentclass[a4paper,11pt]{article}
\usepackage[T1]{fontenc}
\usepackage[utf8]{inputenc}
\usepackage[ngerman]{babel}
\usepackage{lmodern}       % <- skalierbare Schriften (fix für microtype + MiKTeX)
\usepackage{microtype}
\usepackage{amsmath}       % <- für \text in Matheumgebung
\usepackage{csquotes}
\usepackage[hidelinks]{hyperref}
\usepackage[numbers,sort&compress]{natbib}
\usepackage{geometry}
\geometry{margin=2.5cm}

\title{Nichtlineare, spektral aufgelöste geometrische Optik:\\
Kuratierte Literatur (inkl.\ 2020--2024) und Kurzfassungen}
\author{Zusammenstellung für GOAT}
\date{\today}

\begin{document}
\maketitle

\section*{Ziel}
Diese Liste versammelt Kernreferenzen zu (i) nichtlinearer geometrischer Optik (WKB/Eikonal),
(ii) ultrakurzer Pulspropagation und Filamentierung, (iii) Standardmodellen (NLSE/UPPE, Faseroptik) 
sowie (iv) aktuelleren Arbeiten (2020--2024), um einen hybriden, strahlenbasierten, spektral aufgelösten Ansatz wie in \emph{GOAT} einzuordnen.

\section*{Kurzfassungen (klassisch)}

\paragraph{Dumas (2005) \cite{dumas2005nonlinear}}
\textbf{Thema:} Überblick über \emph{nonlinear geometric optics} (NGO). 
\textbf{Kernidee:} Asymptotische WKB-Entwicklungen für schwach nichtlineare Wellengleichungen führen zu nichtlinearen Eikonal- und Transportgleichungen. 
\textbf{Relevanz:} Legt den mathematischen Rahmen, um intensitätsabhängige Phasen (\(n(I)\)) konsistent in einer Strahlenbeschreibung zu führen (Stabilität, Gültigkeit).

\paragraph{Couairon \& Mysyrowicz (2007) \cite{couairon2007femtosecond}}
\textbf{Thema:} Umfassendes Review zu Femtosekunden-Filamentierung in transparenten Medien. 
\textbf{Kernidee:} Konkurrenz von Kerr-Selbstfokussierung, Plasma-Defokussierung, Dispersion und Nichtlinearverlusten; Modelllandschaft von Ray-/Momentenmodellen bis UPPE. 
\textbf{Relevanz:} Zeigt, wie strahlennahe Modelle Selbstfokussierung qualitativ/quantitativ erfassen und wo feldbasierte Beschreibung nötig wird.

\paragraph{Kolesik \& Moloney (2004) \cite{kolesik2004nonlinear}}
\textbf{Thema:} Ableitung und Einsatz unidirektionaler Wellengleichungen (UPPE) aus Maxwell. 
\textbf{Kernidee:} Übergang zwischen Hüllgleichungen und vektorieller Vollfeldbeschreibung; breitbandige, nichtlineare Pulse inkl.\ Dispersion/Raman/Plasma. 
\textbf{Relevanz:} Referenz für die Ground-Truth-Feldmodelle; Benchmark für den strahlenbasierten Hybrid.

\paragraph{Bergé et al. (2007) \cite{berge2007ultrashort}}
\textbf{Thema:} Review zu ultrakurzen Filamenten in schwach ionisierten Medien. 
\textbf{Kernidee:} Dynamik von Selbstfokussierung, Schockbildung, Superkontinuum; Skalenlängen (\(L_{\text{NL}}, L_{\text{disp}}\)) und Stabilitätskriterien. 
\textbf{Relevanz:} Liefert praxisnahe Skalen/Parameter für Schrittweiten und Level-of-Detail.

\paragraph{Brabec \& Krausz (1997) \cite{brabec1997singlecycle}}
\textbf{Thema:} Einhüllengleichung im Single-Cycle-Regime (UPPE-Vorläufer). 
\textbf{Kernidee:} Erste-Ordnung-Propagation längs der Ausbreitung, gültig bis Ein-Zyklus-Pulse; konsistente Breitbandigkeit. 
\textbf{Relevanz:} Fundament für gemischte Operator-Splittings (Fourier in \(t\), Propagation in \(s\)).

\paragraph{Agrawal (2012) \cite{agrawal2012nonlinear}}
\textbf{Thema:} Standardwerk der nichtlinearen Faseroptik (NLSE/GNLSE). 
\textbf{Kernidee:} Systematische Behandlung von Kerr, Raman, Selbst-Steepening, Dispersion; numerische Split-Step-Fourier-Verfahren. 
\textbf{Relevanz:} Blaupause für die Zeitdomänen-Nichtlinearoperatoren in einem strahlenbasierten Split-Step.

\section*{Kurzfassungen (2020--2024)}

\paragraph{Lassonde et al. (2020) \cite{lassonde2020temporalconvolution}}
\textbf{Thema:} Zeitliche Faltung in nichtlinearer Optik und ihre Darstellung im Frequenzraum. 
\textbf{Kernidee:} Nichtlineare Prozesse lassen sich als zeitliche Faltungsoperatoren modellieren, die im Spektrum Multiplikationen werden. 
\textbf{Relevanz:} Direkt anschlussfähig an spektral aufgelöste Modelle wie dein Fourier-basierter Strahlenansatz (SPM/XPM, Raman als Faltung).

\paragraph{Gliserin et al. (2022) \cite{gliserin2022penguin}}
\textbf{Thema:} Vollständige Rekonstruktion des optischen Feldes ultrakurzer Pulse (Amplitude \emph{und} Phase). 
\textbf{Kernidee:} Nichtlineares Gating (PENGUIN) mit unbalancierter Intensität ermöglicht robuste Feldmessung. 
\textbf{Relevanz:} Essenziell zur Validierung eines Modellcodes: gemessene spektrale Phase/\(E(t)\) vs.\ Simulation.

\paragraph{Cai et al. (2020) \cite{cai2020temporalmeasurements}}
\textbf{Thema:} Moderne zeitliche Messverfahren für Ultrakurzpulse (FROG, SPIDER, d-scan, Neuheiten). 
\textbf{Kernidee:} Systematische Einordnung der Messprinzipien, Fehlerquellen und Rekonstruktionsalgorithmen. 
\textbf{Relevanz:} Bietet Referenzen und Protokolle für Vergleichsmessungen deiner Simulationen.

\paragraph{Iordanova (2024) \cite{iordanova2024broadband}}
\textbf{Thema:} Optik breitbandiger Laserimpulse in linearen und nichtlinearen Medien. 
\textbf{Kernidee:} Verbindet Dispersionsanalyse mit nichtlinearen Effekten für breite Spektren; Fokus auf Konzepte statt schwerer Numerik. 
\textbf{Relevanz:} Konzeptioneller Unterbau für spektrale, ray-kompatible Modellierung.

\bibliographystyle{unsrtnat}
\bibliography{refs}
\end{document}
